\documentclass{article}
\usepackage{amsmath}
\usepackage{amssymb}
\usepackage{fullpage}
\usepackage[parfill]{parskip}
\usepackage{pythonhighlight}
\usepackage{scrextend}
\title{Assignment 2 writeup}
\author{Thomas Lu}
\date{}
\begin{document}
\maketitle
\section{Problem 1}
(c) Placeholders are nodes in a TF computation graph whose values are populated at runtime. These nodes are often used to populate training data. Feed dictionaries, meanwhile, are what actually do the populating at runtime; they specify at runtime a mapping of placeholder nodes to actual values.

(e) When \pyth{train_op} is called, we compute during forward propagation (for a batch) predictions $\hat{y} = \text{softmax}(xW + b)$ and the loss $J = CE(y, \hat{y})$, where $CE$ denotes cross-entropy loss. Then we compute during backpropagation the partial gradients $\partial J/\partial W$ and $\partial J/\partial b$ and add a negative multiple of these gradients to our variables $W$ and $b$.

\section{Problem 2}
(a)
\begin{center}
\footnotesize
\begin{tabular}{l|l|l|l}
stack & buffer & new dependency & transition\\
\hline
{[ROOT]} & [I, parsed, this, sentence, correctly] & & Initial Configuration \\
{[ROOT, I]} & [parsed, this, sentence, correctly] & & SHIFT \\
{[ROOT, I, parsed]} & [this, sentence, correctly] & & SHIFT \\
{[ROOT, parsed]} & [this, sentence, correctly] & parsed $\rightarrow$ I & LEFT-ARC \\
{[ROOT, parsed, this]} & [sentence, correctly] & & SHIFT \\
{[ROOT, parsed, this, sentence]} & [correctly] & & SHIFT \\
{[ROOT, parsed, sentence]} & [correctly] & sentence $\rightarrow$ this & LEFT-ARC \\
{[ROOT, parsed]} & [correctly] & parsed $\rightarrow$ sentence & RIGHT-ARC \\
{[ROOT, parsed, correctly]} & [] & & SHIFT \\
{[ROOT, parsed]} & [] & parsed $\rightarrow$ correctly & RIGHT-ARC \\
{[ROOT]} & [] & ROOT $\rightarrow$ parsed & RIGHT-ARC \\
\end{tabular}
\end{center}

(b) A written sentence of $n$ words will require $2n$ steps: each word requires one step to move it from the buffer to the stack, and one to move it from the stack to the dependency tree.

(f) We have
$$ h_i = \mathbb{E}_{drop}[h_{drop}]_i = \gamma (1 - p_{drop}) h_i,$$
so $\gamma = 1/(1 - p_{drop})$.

(g)
\begin{itemize}
\item (i) Momentum basically slows the rate at which we adjust our updates: instead of immediately updating using the current gradient, we instead continue mostly going in the direction of our previous momentum and only assign a partial influence to the current gradient on our next update. This can help us avoid large ``bad" updates when we see a particularly anomalous batch of training data or when we hit a particularly steep gradient wall.
\item (ii) Adam amplifies the movement of parameters with small gradient contributions and reduces that of parameters with large gradient contributions. This can help in cases where we might have large almost-flat regions and small steep walls in our gradient function.
\end{itemize}

(h) The best dev UAS was 88.71, and the tes UAS was 89.19.
\end{document}
